\documentclass[aspectratio=169]{beamer}

\usetheme{Madrid}
\beamertemplatenavigationsymbolsempty

%\usecolortheme{beaver}
% the following colors are modified from beaver
\definecolor{tuna}{rgb}{0.098,0.51,0.996}
\definecolor{thu}{rgb}{0.50,0.36,0.71}

\setbeamercolor{section in toc}{fg=black,bg=white}
\setbeamercolor{item}{fg=tuna,bg=white}
\setbeamercolor{alerted text}{fg=tuna!80!gray}
\setbeamercolor*{palette primary}{fg=tuna!60!black,bg=gray!30!white}
\setbeamercolor*{palette secondary}{fg=tuna!70!black,bg=gray!15!white}
\setbeamercolor*{palette tertiary}{bg=tuna!80!black,fg=gray!10!white}
\setbeamercolor*{palette quaternary}{fg=tuna,bg=gray!5!white}

\setbeamercolor*{sidebar}{fg=tuna,bg=gray!15!white}

\setbeamercolor*{palette sidebar primary}{fg=tuna!10!black}
\setbeamercolor*{palette sidebar secondary}{fg=white}
\setbeamercolor*{palette sidebar tertiary}{fg=tuna!50!black}
\setbeamercolor*{palette sidebar quaternary}{fg=gray!10!white}

%\setbeamercolor*{titlelike}{parent=palette primary}
\setbeamercolor{titlelike}{parent=palette primary,fg=tuna}
\setbeamercolor{frametitle}{bg=gray!10!white}
\setbeamercolor{frametitle right}{bg=gray!60!white}

\setbeamercolor*{separation line}{}
\setbeamercolor*{fine separation line}{}

\setbeamertemplate{sections/subsections in toc}[square]
\setbeamertemplate{items}[square]

% https://tex.stackexchange.com/questions/248131/delete-institute-space-in-title-page-beamer-class
\defbeamertemplate{title page}{noinstitute}[1][]
{
  \vbox{}
  \vfill
  \begingroup
    \centering
    \begin{beamercolorbox}[sep=8pt,center,#1]{title}
      \usebeamerfont{title}\inserttitle\par%
      \ifx\insertsubtitle\@empty%
      \else%
        \vskip0.25em%
        {\usebeamerfont{subtitle}\usebeamercolor[fg]{subtitle}\insertsubtitle\par}%
      \fi%     
    \end{beamercolorbox}%
    \vskip1em\par
    \begin{beamercolorbox}[sep=8pt,center,#1]{author}
      \usebeamerfont{author}\insertauthor
    \end{beamercolorbox}
    \begin{beamercolorbox}[sep=8pt,center,#1]{date}
      \usebeamerfont{date}\insertdate
    \end{beamercolorbox}\vskip0.5em
    {\usebeamercolor[fg]{titlegraphic}\inserttitlegraphic\par}
  \endgroup
  \vfill
}

\makeatletter
\setbeamertemplate{title page}[noinstitute][colsep=-4bp,rounded=true,shadow=\beamer@themerounded@shadow]
\makeatother

\AtBeginSection[]{
  \begin{frame}
  \vfill
  \centering
  \begin{beamercolorbox}[sep=8pt,center,shadow=true,rounded=true]{title}
    \usebeamerfont{title}\insertsectionhead\par%
  \end{beamercolorbox}
  \vfill
  \end{frame}
}

\usepackage{tikz}
\usetikzlibrary{calc}
\usetikzlibrary{arrows}
\usetikzlibrary{shapes}
\usepackage{hyperref}
\usepackage[UTF8]{ctex}

\newcommand{\T}[1]{\texttt{#1}}

\title[Learn Git]{Learn Git The Not So Super Hard Way\footnote{Credit to https://github.com/b1f6c1c4/learn-git-the-super-hard-way}}
\author[zenithal]{Zenithal \\ Woshiluo Luo As translator}
\date{2024-11-24}

\begin{document}

\begin{frame}
\titlepage
\end{frame}

\begin{frame}{为什么需要这个}
  \begin{itemize}
    \item<1-> 学习 Git 太痛苦了\begin{itemize}
      \item 太多概念 (commit, branch, stage, index)
      \item 太多命令 (clone, pull, push)
    \end{itemize}
    \item<2-> 状态模型太过复杂\begin{itemize}
      \item 你常常不知道自己在哪个状态下。
      \item 冲突! Help me ERIN!
    \end{itemize}
    \item<3-> 知道命令是远远不够的\begin{itemize}
      \item 你常常不知道发生了什么
      \item 让我们来看一些更基本的组件
    \end{itemize}
    \item<4-> 最困难的路,最简单的方法。\begin{itemize}
      \item 你对文件做了改动,你知道发生了什么。
    \end{itemize}
  \end{itemize}
\end{frame}

\section{init}
\begin{frame}{git repo structure}
  \begin{itemize}
    \item git repo\begin{itemize}
      \item 通常在 the \T{.git}
      \item 通常包含 HEAD, config
    \end{itemize}
    \item worktree\begin{itemize}
      \item 文件们
      \item 通常包含 README.md, main.c, main.h
      \item worktree 只是 git repo 的一个 checkout
      \item 你可以从 git repo 中重新构建 worktree
      \item Git repo 是一切的基石,worktree 并不是
    \end{itemize}
  \end{itemize}
\end{frame}

\begin{frame}{git init}
  \begin{itemize}
    \item \T{mkdir .git}
    \item \T{mkdir .git/objects}\begin{itemize}
      \item 必须项
    \end{itemize}
    \item \T{mkdir .git/refs}\begin{itemize}
      \item 必须项
    \end{itemize}
    \item \T{echo 'ref: refs/heads/master' > .git/HEAD}\begin{itemize}
      \item 建立 {HEAD} ref
      \item HEAD 指向 \T{.git/refs/heads/master} (尽管现在还不存在)
      \item 即: \T{refs/heads/main}
    \end{itemize}
    \item config, hooks, info, etc 并非必要
    \item 现在你可以使用 \T{git status} 来检查状态
  \end{itemize}
\end{frame}

\section{objects}
\begin{frame}{objects}
  \begin{itemize}
    \item<1-> 你已经创建了 \T{.git/objects}, 那么什么是 objects
    \item<2-> 四种 objects\begin{itemize}
      \item<3-> blob: 文件内容
      \item<4-> tree: 文件夹\begin{itemize}
        \item 注记: 文件系统中的文件夹
        \item 文件名 (先于 blob 储存!)
        \item 树和 blobs 的 hash (文件夹结构!)
      \end{itemize}
      \item<5-> commit: 根文件夹的一个状态\begin{itemize}
        \item 包含一个指定的树
        \item 祖先(们): 其他 commit 
        \item author/committer/commit message: meta data
      \end{itemize}
      \item<6-> tag: 不是本课件的讨论范围
    \end{itemize}
  \end{itemize}
\end{frame}

\begin{frame}[fragile]{blob}
  \begin{itemize}
    \item<1-> blob: 文件内容
    \item<2-> \T{echo 'hello' | git hash-object -t blob --stdin -w}\begin{itemize}
      \item 写入一个内容为 \T{'hello'} 的 blob/file
      \item 得到对于一个类型为 \T{blob} 的 object,从 \T{stdin} 输入并 写入(\T{w}rite ) object database
      \item 输出: \T{ce013625030ba8dba906f756967f9e9ca394464a}, the hash of the object
    \end{itemize}
    \item<3-> \T{cat .git/objects/ce/013625030ba8dba906f756967f9e9ca394464a}\begin{itemize}
      \item 输出: \T{xKOR0cH}, hello 的压缩后数据
      \item 注意该文件的路径!
    \end{itemize}
    \item<4-> 查看真实内容 \begin{verbatim}
    $ printf '\x1f\x8b\x08\x00\x00\x00\x00\x00' \
    | cat - .git/objects/ce/013625030ba8dba906f756967f9e9ca394464a \
    | gunzip -dc 2>/dev/null | xxd
    # 00000000: 626c 6f62 2036 0068 656c 6c6f 0a         blob 6.hello.
    \end{verbatim}
  \end{itemize}
\end{frame}

\begin{frame}{blob (cont'd)}
  \begin{itemize}
    \item 使用这些直接命令很痛苦?当然我们有更高级的指令
    \item \T{git cat-file blob ce01}\begin{itemize}
      \item Output: \T{hello}
    \end{itemize}
    \item \T{git show ce01}\begin{itemize}
      \item Output: \T{hello}
    \end{itemize}
  \end{itemize}
\end{frame}

\begin{frame}[fragile]{tree}
  \begin{itemize}
    \item<1-> tree: 文件夹
    \item<2-> 创建一棵 tree\begin{verbatim}
(printf '100644 name.ext\x00';
echo '0: ce013625030ba8dba906f756967f9e9ca394464a' | xxd -rp -c 256;
printf '100755 name2.ext\x00';
echo '0: ce013625030ba8dba906f756967f9e9ca394464a' | xxd -rp -c 256) \
| git hash-object -t tree --stdin -w
# 58417991a0e30203e7e9b938f62a9a6f9ce10a9a
\end{verbatim}
    \item<3-> 你也可以(另一种形式)\begin{verbatim}
git mktree --missing <<EOF
100644 blob ce013625030ba8dba906f756967f9e9ca394464a$(printf '\t')name.ext
100755 blob ce013625030ba8dba906f756967f9e9ca394464a$(printf '\t')name2.ext
EOF
# 58417991a0e30203e7e9b938f62a9a6f9ce10a9a
\end{verbatim}
  \end{itemize}
\end{frame}

\begin{frame}[fragile]{tree (cont'd)}
  \begin{itemize}
    \item<1-> 直接查看文件内容\begin{verbatim}
printf '\x1f\x8b\x08\x00\x00\x00\x00\x00' \
| cat - .git/objects/58/417991a0e30203e7e9b938f62a9a6f9ce10a9a \
| gunzip -dc 2>/dev/null | xxd
\end{verbatim}
    \item<2-> \T{git cat-file tree 5841 | xxd}
    \item<3-> \T{git ls-tree 5841} (和上文 mktree 相对)
    \item<4-> \T{git show 5841} (一个更简单的版本)
  \end{itemize}
\end{frame}

\begin{frame}[fragile]{commit}
  直接创建文件\begin{verbatim}
git hash-object -t commit --stdin -w <<EOF
tree 58417991a0e30203e7e9b938f62a9a6f9ce10a9a
author b1f6c1c4 <b1f6c1c4@gmail.com> 1514736000 +0800
committer b1f6c1c4 <b1f6c1c4@gmail.com> 1514736000 +0800

The commit message
May have multiple
lines!
EOF
# d4dafde7cd9248ef94c0400983d51122099d312a
\end{verbatim}
\end{frame}

\begin{frame}[fragile]{commit (cont'd)}
  或者一个更高级的指令\begin{verbatim}
GIT_AUTHOR_NAME=b1f6c1c4 \
GIT_AUTHOR_EMAIL=b1f6c1c4@gmail.com \
GIT_AUTHOR_DATE='1600000000 +0800' \
GIT_COMMITTER_NAME=b1f6c1c4 \
GIT_COMMITTER_EMAIL=b1f6c1c4@gmail.com \
GIT_COMMITTER_DATE='1600000000 +0800' \
git commit-tree 5841 -p d4da <<EOF
Message may be read
from stdin
or by the option '-m'
EOF
# efd4f82f6151bd20b167794bc57c66bbf82ce7dd
\end{verbatim}
  这也是为什么你需要执行 \T{git config --global user.email} 和 \T{user.name}
\end{frame}

\begin{frame}[fragile]{commit (cont'd)}
  \begin{itemize}
    \item<1-> 直接观察文件\begin{verbatim}
printf '\x1f\x8b\x08\x00\x00\x00\x00\x00' \
| cat - ./objects/ef/d4f82f6151bd20b167794bc57c66bbf82ce7dd \
| gunzip -dc 2>/dev/null | xxd
\end{verbatim}
    \item<2-> \T{git cat-file commit efd4}
    \item<3-> \T{git show efd4~} (一种更简单的方式,以 diff 形式展示)
    \item<3-> 注: commits 是快照, 并非 diffs/patchs\footnote{https://github.blog/2020-12-17-commits-are-snapshots-not-diffs/}
  \end{itemize}
\end{frame}

\begin{frame}[fragile]{Lucky commit}
  \begin{itemize}
    \item Hash 太无聊了!
    \item 试试 lucky commit!\footnote{https://github.com/not-an-aardvark/lucky-commit}
    \item \begin{verbatim}
$ git log
1f6383a Some commit
$ lucky_commit
$ git log
0000000 Some commit
\end{verbatim}
    \item 注意前一页的 commit msg,我们可以通过控制他来得到了一个 lucky hash
  \end{itemize}
\end{frame}

\section{ref}
\begin{frame}{ref}
  \begin{itemize}
    \item ref 是一个方便的,指向一个特定 commit 或者其他 ref 的引用
    \item 存在 \T{.git/ref}
    \item 两种 ref\begin{itemize}
      \item 直接 ref
	  \item 间接 ref, e.g. \T{HEAD} (常见情况)
    \end{itemize}
    \item 今天将会介绍两个常见 ref\begin{itemize}
      \item heads: 本地分支
      \item remotes: 远端分支
    \end{itemize}
  \end{itemize}
\end{frame}

\begin{frame}[fragile]{local branch and direct ref}
  \begin{itemize}
    \item<1-> 直接创建文件 (由于没有 reflog 所以不推荐)\begin{verbatim}
mkdir -p .git/refs/heads/
echo d4dafde7cd9248ef94c0400983d51122099d312a > .git/refs/heads/br1
\end{verbatim}
    \item<2-> 下面的命令会在 \T{.git/log/refs/heads/br1} 留下 reflog
    \item<2-> \T{git update-ref --no-deref -m 'Reason for update' refs/heads/br1 d4da}
    \item<2-> \T{git branch -f br1 d4da}
  \end{itemize}
\end{frame}

\begin{frame}{关于 reflog}
  \begin{itemize}
    \item 记录你的 ref 的一切变化
    \item 在你一不小心 swich 到其他地方时很有用\begin{itemize}
      \item \T{git rebase master}
      \item \T{git checkout -B master origin/master}
      \item 之后你由于某种原因想要 switch 到之前的 tree
      \item reflog 会给出一个 ref \textbf{曾经的} commit
    \end{itemize}
    \item 一个栗子: lots of reflogs
  \end{itemize}
\end{frame}

\begin{frame}{indirect ref}
  \begin{itemize}
    \item 还记得你初始化 git repo 的命令吗?\begin{itemize}
      \item \T{echo 'ref: refs/heads/master' > .git/HEAD}
    \end{itemize}
    \item 这种格式就是 indirect ref
  \end{itemize}
\end{frame}

\section{index}
\begin{frame}{index}
  \begin{figure}
    \centering
    \includegraphics[width=0.7\textwidth]{img/lifecycle.png}
  \end{figure}
  \begin{itemize}
    % stage figure here
    \item index 储存你想要 在 \T{git commit} commited 的内容
    \item 文件是 \T{.git/index}
    \item 通常叫在 index 中的文件为 staged (如上文所说)
    \item 一个复杂的数据库
    \item 包含很多东西, 如 filename, mode, hash, mtime, 等
  \end{itemize}
\end{frame}

\begin{frame}{操纵 index}
  \begin{itemize}
    \item<1-> 这很困难
    \item<1-> 我们可以学习一些常规情况
    \item<1-> \T{git add} 将内容存入 index
    \item<1-> 将他们标记为将被 commit
    \item<2-> 1. \T{git add; git status}\begin{itemize}
      \item 那些已经被你加入 index 的文件将被 commit
    \end{itemize}
    \item<3-> 2. \T{git add; modify; git status}\begin{itemize}
      \item 你已经 added 的文件内容将被 commit
	  \item 新的文件内容尚未被 add,并不在 index 里
      \item \T{modify} 将不会被 commit
    \end{itemize}
    \item<4-> 3. \T{git add; rm; git status; git restore}\begin{itemize}
      \item 尽管文件已经被删除,但是他仍在 \T{index} 中有一份拷贝
      \item 如果你不慎 \T{rm -rf *}, 你可以恢复你的文件! 
    \end{itemize}
  \end{itemize}
\end{frame}

\section{switch/checkout}
\begin{frame}{switch/checkout}
  \begin{itemize}
    \item 回忆一下, \T{.git/HEAD} 是一个 ref
    \item 这是你 worktree 的 ref
    \item 回忆一下,你的 worktree 是你实际看到的内容
    \item 我们可以通过操纵 HEAD 来修改你的 worktree 
  \end{itemize}
\end{frame}

\begin{frame}{switch/checkout (cont'd)}
  \begin{itemize}
    \item<1-> 最著名的: \T{git checkout master}\begin{itemize}
      \item 让 HEAD 指向 \T{refs/heads/master}
	  \item 然后 checkout 你 worktree 的内容
      \item 这就是他为什么叫 checkout
      \item 其实是一个就语法了,现在推荐使用 switch
      \item \T{git switch master}
    \end{itemize}
    \item<2-> 也很著名的: \T{git reset --hard HEAD$\sim$1}\begin{itemize}
      \item 将 HEAD 指向 HEAD$\sim$1 (the former commit of HEAD)
	  \item checkout 你 worktree 的内容
      \item 注: 还有 \T{reset --soft/--mixed}, 请自行研究
    \end{itemize}
  \end{itemize}
\end{frame}

\section{pull/clone/push}
\begin{frame}{remotes}
  \begin{itemize}
    \item 我们已经提到过 \T{.git/refs/remotes}
    \item 既然我们有 local ref(branch), 我们也会有 remote ref(branch)
    \item 没有 remote branch, 算什么分布式版本控制系统(distributed version control system)
    \item 如果同步两端呢?
    \item pull commit 从 remote 到 local
    \item push commit 从local 到 remote
    \item 所以 commit 的概念是非常有用的!
  \end{itemize}
\end{frame}

\begin{frame}{config remote}
  \begin{itemize}
    \item 如果你还没有 remote,那么添加 remote 是必要的
    \item 编辑 \T{.git/config} 来添加
    \item 或者 \T{git remote add origin git@github.com:xxx/yyy}\begin{itemize}
      \item \T{origin} 是一个惯用名,你可以取一个自己喜欢的名字。
      \item 你可以有多个 remote
    \end{itemize}
    \item Demo of my repo
  \end{itemize}
\end{frame}

\begin{frame}{fetch remote}
  \begin{itemize}
    \item<1-> \T{git fetch origin master}\begin{itemize}
      \item 从 \T{origin} 获取 \T{master} ref 
      \item 你现在可以看看 \T{.git/refs/remotes/origin/master} 了
    \end{itemize}
    \item<2-> \T{git pull origin master}\begin{itemize}
      \item 不同于 \T{git fetch}, 他会尝试更新你的 local ref
      \item 更新 \T{.git/refs/remotes/origin/master}
      \item 并同时更新 \T{.git/refs/heads/master} 
      \item 之间的关于会在 \T{.git/config} 中记录
    \end{itemize}
    \item<3-> \T{git pull}\begin{itemize}
      \item 上个命令的缩写,会根据 \T{.git/config} 中的记录执行
    \end{itemize}
    \item<4-> \T{git clone}\begin{itemize}
      \item 事实上是下面命令的缩写
      \item \T{git init}
      \item \T{git remote add}
      \item \T{git pull}
    \end{itemize}
  \end{itemize}
\end{frame}

\begin{frame}{push remote}
  \begin{itemize}
    \item<1-> \T{git push origin master}\begin{itemize}
      \item 将 local branch master 同步到 remote branch master
    \end{itemize}
    \item<2-> \T{git push}\begin{itemize}
      \item 上个命令的缩写,遵循 \T{.git/config}
    \end{itemize}
    \item<3-> 一个新的 branch 并执行 \T{git push -u origin :new-branch}\begin{itemize}
      \item 在 remote 添加一个新的 ref
      \item 同时设为该 branch 的 upstream
      \item 现在可以看看你的 \T{.git/config}
    \end{itemize}
  \end{itemize}
\end{frame}

\section{merge}
\begin{frame}{merge}
  \begin{itemize}
    \item 你有 commits 了,你有 refs 了
    \item 你如何 merge(合并) refs/branches 到一起呢?
    \item 会议: branch 指向一个 commit, 一个 commit 包含一个指定的 tree
    \item 即我们需要 merge tree, 也就是我们需要先 merge blob 
    \item 如何 merge blob?
  \end{itemize}
\end{frame}

\begin{frame}{two way merge}
  \begin{itemize}
    \item Two way 意味着算法只关心两个部分(自己和其他)
    \item 以 \T{chapter6.md} 文件为例
    \item \T{fileB} 和 \T{fileC} 的 two way merge \begin{itemize}
      \item fileC 在第一行没有 B 并在最后一行有 C
	  \item fileB 有第一行的 B 且最后一行没有 C
      \item 不知道如何合并,终止
    \end{itemize}
    \item 这并不实用
  \end{itemize}
\end{frame}

\begin{frame}[fragile]{three way merge}
  \begin{itemize}
    \item Three way merge 意味着算法关心三个部分 (base, our and their)
	\item \T{fileB}, \T{fileC} 和 以 \T{fileA} 为 base 的 three way merge\begin{itemize}
      \item 相较于 \T{fileA}, \T{fileB} 在第一行加了 B
      \item 相较于 \T{fileA}, \T{fileC} 在最后一行加了 C
      \item 变化并无冲突
      \item \T{git merge-file --stdout <our> <base> <their>}
      \item \T{git merge-file --stdout fileC fileA fileB}\begin{verbatim}
lineBB
...some stuff...
lineCC
\end{verbatim}
    \end{itemize}
  \end{itemize}
\end{frame}

\begin{frame}[fragile]{three way merge (cont'd)}
  \begin{itemize}
    \item 那么如果两段都对同一行有修改呢?冲突!
    \item 通常需要手动干涉
    \item 栗子 \T{git merge-file --stdout fileD fileA fileB}\begin{itemize}
      \item 相较于 \T{fileA}, \T{fileD} 在第一行加了 D 
      \item 相较于 \T{fileA}, \T{fileB} 在第一行加了 B 
      \item Output\begin{verbatim}
<<<<<<< fileD
lineBD
=======
lineBB
>>>>>>> fileB
...some stuff...
lineC
\end{verbatim}
    \end{itemize}
  \end{itemize}
\end{frame}

\begin{frame}[fragile]{如何解决冲突}
  \begin{itemize}
    \item 移除 healper line
    \item 保留该保留的内容\begin{verbatim}
lineBBD
...some stuff...
lineC
\end{verbatim}
    \item 或者,如果你清楚你在做什么\begin{itemize}
      \item \T{git merge-file --ours --stdout fileD fileA fileB}
      \item 保留我们的(our)的修改,移除他们的(their)的修改
      \item \T{git merge-file --theirs --stdout fileD fileA fileB}
      \item 保留他们的修改,移除我们的修改
      \item \T{git merge-file --union --stdout fileD fileA fileB}
      \item 保留两段的修改,并连接到一起
    \end{itemize}
  \end{itemize}
\end{frame}

\end{document}

% vim: nospell
